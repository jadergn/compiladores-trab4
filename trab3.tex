\documentclass[a4paper,10pt]{article}
\usepackage[utf8]{inputenc}

%opening
\title{3º Trabalho de Compiladores - Analisador Semântico}
\author{Jader Gomes Nascimento e Fernando Guimarães Pinheiro}

\begin{document}

\maketitle

\begin{abstract}
Documentação referente à terceira parte do trabalho de compiladores, que consiste em um analisador semântico.
\end{abstract}

\section{Introdução}
O objetivo deste trabalho é estabelecer um analisador semântico para a linguagem GPortugol. O analisador semântico é a terceira parte
de um compilador, e sua função é verificar se as variáveis utilizadas foram declaradas, se há variáveis redeclaradas, se há variáveis declaradas e não utilizadas, se os tipos associados às variáveis e ao valor associado são compatíveis, se o número de argumentos (aridade) de uma função ou procedimento está correto, se o tipo associado ao valor de retorno de uma função está correto e se os tipos associados aos argumentos de uma função ou procedimento estão corretos.


\section{Analizador semântico}
Para o desenvolvimento do trabalho foram utilizados dois TAD's: tabela hash e pilha para aulixio.

\section{Tabela Hash}
Principal funcionalidade do trabalho. Seu tamanho é 97. É implementado para isso o TAD lista, que é uma lista genérica, ou seja, é utilizada a mesma estrutura para variáveis e funções.
A tabela hash trata-se de um de vetor com tamanho 97, onde cada posição do vetor é uma lista.

Além dos TAD's citados acima outros dois TAD's são usados: Variável e Função. Neles são guardadas as informações (e valores no caso das variáveis) que foram declaradas em cada programa.

\section{Pilha}
Trata-se de uma pilha normal e é usado para verificações de tipos em expressões.
Apesar de não ter tamanho limite de itens são usados no máximo dois. A cada identificador ou valor primitivo encontrado nós obtemos o tipo desse termo e inserimos na pilha e fazemos uma verificação. Sempre que o segundo item é inserido é verificado a compatibilidade dos dois itens existentes e caso isso não ocorra é retornado um erro. Caso contrario fica apenas um item com o tipo compatível (no caso de operações com inteiro e real).

Poderia ter sido usado de uma maneira diferente. No caso da declarações de variáveis onde so obtemos o tipo das variáveis depois ter termos por todas elas.

\section{Observações}
  \begin{itemize}
    \item Pequenas alterações foram feitas no analisador sintático
    \item Foi adicionado um novo token na linguagem que indica o fim da função: fim-função.
    \item Funções obrigatoriamente devem ser declaradas antes do bloco inicio principal e antes do bloco de declaração de variáveis.
    \item Tipo booleano não aceito.
    \item É feito verificação de tipos nas expressões com coersão se necessário.
    \item Expressão dentro de expressões não é aceito.
  \end{itemize}

\section{Exemplos}
Nos exemplos, foi explorado diversas situações para os diversos tipos de usuários. Foram criadas 10 situações, 5 situações corretas e 5
situações erradas, tentando utilizar todas as funções que foram definadas.

\end{document}
